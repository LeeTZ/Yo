% @Author: LeeTZ
% @Date:   2015-10-14 14:28:14
% @Last Modified by:   LeeTZ
% @Last Modified time: 2015-10-14 15:40:30

\documentclass[12pt]{article}
\usepackage[T1]{fontenc}     
\usepackage[utf8]{inputenc}  % Accents codés dans la fonte
\usepackage[frenchb]{babel}  % Les traductions françaises

\usepackage{amsmath}         % Les maths de base
\usepackage[svgnames]{xcolor}% Pour les besoins de PythonTeX
\usepackage{geometry}        % Gestion des dimensions des pages
\usepackage{graphicx}        % Gestion des inclusions graphiques
\usepackage{listings}

\usepackage{subcaption} 
\usepackage{color}
\usepackage{xcolor}
\DeclareCaptionFont{black}{\ttfamily\color{black}}
\usepackage{hhline}

\usepackage[font=black]{caption}
\DeclareCaptionFormat{listing}{\colorbox{gray!22}{\parbox{\textwidth}{#1#2#3}}}
\captionsetup[lstlisting]{format=listing,labelfont=black,textfont=black,skip=-2pt}
\lstset{
    xleftmargin=3.5pt,
    xrightmargin =-2.1pt,
    aboveskip= 2pt,
    basicstyle=\ttfamily\small,
    commentstyle=\mdseries\color{gray},
    frame=lbr,
    language = Python,
    keywordstyle=\bfseries\color{blue},
    escapeinside={(*@}{@*)},
    rulecolor=\color{gray!50}
 }
 
 \lstset{
    emph={
        downto, func, ret, \&\&, ||%
    } ,emphstyle={\bfseries\color{blue}}
 }
 
\renewcommand{\lstlistingname}{Example}


% Un raccourci pour composer les unités en caractères droits
\newcommand{\U}[1]{~\mathrm{#1}}

% Présentation de l'abstract pour la problématique
\usepackage[runin]{abstract}

\usepackage{enumitem}
\renewcommand{\labelitemi}{$\vcenter{\hbox{\tiny$\bullet$}}$}


% Titre et auteurs du document
\title{\textit{Yo}: Language Reference Manual}
\author{Mengqing Wang, Munan Cheng, Tiezheng Li, Yufei Ou}
\date{\{mw3061,mc4081,tl2693,yo2265\}@columbia.edu}

\begin{document}

\maketitle
\tableofcontents

\section{Introduction}
\subsection{What is Yo}
\textit{Yo} is a user-friendly programming language for movie production. We offer the fastest and most efficient non-linear video editing and analyzing. Users can produce videos from varieties of sources such as images or existing video clips and apply system- or user-defined functions to perform seamless video editing such as clip construction, duration adjustment, subtitle burning. Besides, \textit{Yo} provides strong self-defined libraries for digital video analysis, such as sentimental analysis and pattern recognition etc. In this light, \textit{Yo}'s objective is to facilitate analytical editing on videos and less human effort needs to be involved.

\subsection{Frame, Clip, Layer}

The concepts of \textit{frame}, \textit{clip} and \textit{layer} are as illustrated. These concepts are recursively defined but can be simplified as follows. A \textit{frame} is seen as one of a sequence still images which compose a \textit{clip}. It can be constructed directly from an image stored on the hard disk or an extract from an existing \textit{clip}. Once a \textit{clip} is assembled from a series of frames at a certain frame rate (usually 24 frames per second), it can be exported as the final product, or to be layered with other \textit{clip}s to form a new \textit{clip}.

Below we show the common operations on these elements.
\begin{table}[h!]
  \begin{center}
    \begin{tabular}{c|l}
      \text{Frames} & \begin{tabular}[x]{@{}l@{}}Contruct a frame from an image \\ Extract frames from clips \\ Modify a frame (e.g. adjust image color)\end{tabular}\\
      \hline
      \text{Clips} &\begin{tabular}[x]{@{}l@{}} Construct a clip from frames \\ Trim, concatenate clips  \\ Adjust playing speed \\ Layer multiple clips and form a new one \\ Arrange inner-clips on the time line \end{tabular}\\
      \hline
      \text{Layers} & \textit{Yo} does not support operations on conceptual layers
    \end{tabular}
    \caption{Operations for frames, layers and clips}
    \label{tab:table1}
  \end{center}
\end{table}
\section{Syntax Notations}
describe the syntax notations used in this reference manual.\\
e.g. How to note identifier or functions here, the usage of italic or bold format, the usage of \{ \}.
\section{Lexical Conventions}
\subsection{Comments}
single line\\
multi lines\\
nest or not\\
\subsection{Identifiers}
the naming convetions of identifier
\subsection{Keywords}
list all the keywords reserved
\subsection{Constants}
integer, real number, string constants
\subsection{Operators}
An operator is a special token that performs an operation, such as addition or subtraction, on either one, two, or three operands. Full coverage of operators can be found in a later chapter. 
\subsection{Separators}
A separator separates tokens. White space (see next section) is a separator, but it is not a token. The other separators are all single-character tokens themselves: \\
( ) [ ]  , \\
\subsection{White space}
blank and tab \\
here we use indent to saparate blocks\\

\section{Data Types}

\subsection{Type Inference}
we have many data types, see below\\
but we do not have to explictly declare the type of a variable\\

\subsection{Primitive Data Types}
\subsubsection{Integer}
\subsubsection{Real number Double}
\subsubsection{Boolean}
\subsubsection{String}


\subsection{Non-Primitive Data Types}
\subsubsection{array}
define, declaring, member access, member functions here.
\subsubsection{tuple}
\subsubsection{struct}

\subsection{Built-in Types}
\subsubsection{Frame}
\subsubsection{Clip}
layer acts as a member of clip 

\section{Expressions}
\subsection{Primary expressions}
\subsection{Assignment operators}
\subsection{Arithmetic operators}
\subsection{Comparison operators}
\subsection{Logical operators}
\subsection{Clip operators}
\subsection{Member Access operators}
\subsection{Operator Precedence}


\section{Statements}
\subsection{Expressions}
\subsection{continue}
\subsection{break}
\subsection{ret}
\subsection{if}
\subsection{while}
\subsection{for}


\section{Functions}
\subsection{Definition}
\subsection{Calling}
\subsection{Lambdas}

\section{Library Functions}
\subsection{map}
\subsection{foldleft}
\subsection{filter}
\subsection{ReadClip}
And those built-in functions like readclip,log..\\ 
each one as a subsection.
\section{Program Structure and Scope}

\subsection{Program Structure}
a basic view of program Structure.\\
\subsection{Scope}
the scope and lifetime of varialbes\\

\section{Sample Program}
paste a sample program with well-written comments here\\
\end{document}