\documentclass[12pt]{article}
\usepackage[T1]{fontenc}     
\usepackage[utf8]{inputenc}  % Accents codés dans la fonte
\usepackage[frenchb]{babel}  % Les traductions françaises

\usepackage{amsmath}         % Les maths de base
\usepackage[svgnames]{xcolor}% Pour les besoins de PythonTeX
\usepackage{geometry}        % Gestion des dimensions des pages
\usepackage{graphicx}        % Gestion des inclusions graphiques
\usepackage{listings}

\usepackage{subcaption} 
\usepackage{color}
\usepackage{xcolor}
\DeclareCaptionFont{black}{\ttfamily\color{black}}
\usepackage{hhline}

\usepackage[font=black]{caption}
\DeclareCaptionFormat{listing}{\colorbox{gray!22}{\parbox{\textwidth}{#1#2#3}}}
\captionsetup[lstlisting]{format=listing,labelfont=black,textfont=black,skip=-2pt}
\lstset{
    xleftmargin=3.5pt,
    xrightmargin =-2.1pt,
    aboveskip= 2pt,
    basicstyle=\ttfamily\small,
    commentstyle=\mdseries\color{gray},
    frame=lbr,
    language = Python,
    keywordstyle=\bfseries\color{blue},
    escapeinside={(*@}{@*)},
    rulecolor=\color{gray!50}
 }
 
 \lstset{
 	emph={
    	downto, func, ret, \&\&, ||%
    } ,emphstyle={\bfseries\color{blue}}
 }
 
\renewcommand{\lstlistingname}{Example}


% Un raccourci pour composer les unités en caractères droits
\newcommand{\U}[1]{~\mathrm{#1}}

% Présentation de l'abstract pour la problématique
\usepackage[runin]{abstract}

\usepackage{enumitem}
\renewcommand{\labelitemi}{$\vcenter{\hbox{\tiny$\bullet$}}$}


% Titre et auteurs du document
\title{\textit{Yo}: A Video Analytical Editing Programming Language}
\author{Mengqing Wang, Munan Cheng, Tiezheng Li, Yufei Ou}
\date{\{mw3061,mc4081,tl2693,yo2265\}@columbia.edu}

\begin{document}

\maketitle

\section{Introduction}
\textit{Yo} is a user-friendly programming language for movie production. We offer the fastest and most efficient non-linear video editing and analyzing. Users can produce videos from varieties of sources such as images or existing video clips and apply system- or user-defined functions to perform seamless video editing such as clip construction, duration adjustment, subtitle burning. Besides, \textit{Yo} provides strong self-defined libraries for digital video analysis, such as sentimental analysis and pattern recognition etc. In this light, \textit{Yo}'s objective is to facilitate analytical editing on videos and less human effort needs to be involved.



\section{Frame, Clip, Layer}
\begin{figure}[ht]
	\centering
	\includegraphics[width=\textwidth]{concepts.pdf}
	\caption{Frame, Clip and Layer}
	\label{fig:concepts}
\end{figure}

The concepts of \textit{frame}, \textit{clip} and \textit{layer} are as illustrated in Figure \ref{fig:concepts}. These concepts are recursively defined but can be simplified as follows. A \textit{frame} is seen as one of a sequence still images which compose a \textit{clip}. It can be constructed directly from an image stored on the hard disk or an extract from an existing \textit{clip}. Once a \textit{clip} is assembled from a series of frames at a certain frame rate (usually 24 frames per second), it can be exported as the final product, or to be layered with other \textit{clip}s to form a new \textit{clip}.

Below we show the common operations on these elements.
\begin{table}[h!]
  \begin{center}
    \begin{tabular}{c|l}
      \text{Frames} & \begin{tabular}[x]{@{}l@{}}Contruct a frame from an image \\ Extract frames from clips \\ Modify a frame (e.g. adjust image color)\end{tabular}\\
      \hline
      \text{Clips} &\begin{tabular}[x]{@{}l@{}} Construct a clip from frames \\ Trim, concatenate clips  \\ Adjust playing speed \\ Layer multiple clips and form a new one \\ Arrange inner-clips on the time line \end{tabular}\\
      \hline
      \text{Layers} & \textit{Yo} does not support operations on conceptual layers
    \end{tabular}
    \caption{Operations for frames, layers and clips}
    \label{tab:table1}
  \end{center}
\end{table}
\pagebreak


\section{Features}
To reduce the learning curve for new users, \textit{Yo} scripts borrows much grammar from Python and C++. The user code would be compiled into C++ code (to be compiled by a C++ compiler) and executed utilizing a collection of C++ libraries such as \texttt{libopenshot}\footnote{https://launchpad.net/libopenshot}. 

\textit{Yo} is planned to support the following language features:
\begin{enumerate}
\item Automatic garbage collection and easy interpolation with existing C++ code and libraries.
\item Maximum code cleanliness: indent blocking, newline instead of colon between statements.%: Syntax is a lot more cleaner and it is easy for compilers to identify tabs instead of self-defined characters.
\item Anonymous function.
\item Functional syntax sugar: with built-in functions such as \texttt{map, filter, lapply}, users can define powerful inline expressions.%anonymous functions from these higher-order utilities. 
\item Object-oriented programming.%: Object-oriented paradigm has great advantages including code reuse and encapsulation.
\item Deep optimization by compiler.
\end{enumerate}

\section{Use Case}

Now we show how \textit{Yo} facilitates movie editing with a couple of examples.

\subsection{Storyboard}
The first example involves concatenating a collection of images into a clip, joining it with a series of existing clips, and showing a subtitle on the most front layer.
  
\begin{lstlisting}[caption=Arrange clips and add subtitles]
# Read all images in directory "wd"
frames = [readFrame(f) for f in wd if f.endwith(".png")]

# Turn each Frame into a Clip of one second, and concatenate
# them into a new Clip called "clip_f"
clip_f = lapply(+, Clip(), [Clip(fm, dur=1.0) for fm in frames])

# Read all videos in directory "wd"
clips = [readVideo(f) for f in wd if f.endwith(".avi")]

# Join a part of clips[0] (4.5s to 12.5s) and clips[1] 
# (starting from 3.5s till end) to "clip_f"
clip_v = clip_f + clips[0](4.5:12.5) + clips[1](3.5:)

# Add a subtitle above "clip_f" at 7.0s which lasts 3.0s
clip_st = clip_v ^ Subtitle('Yo, world!', duration=3.0) @ 7.0
\end{lstlisting}

\subsection{Effects}
Next we apply quick color corrections to a part of the clip.
\begin{lstlisting}[caption=Add effects to frames]
# Define a function that recolors all pixels RGB(>140,?,<50)
func changeColor(fm)
  for row in fm.pixels
    for p in row
      if p.r > 140 && p.b < 50
        p.r = 200
        p.b = 20
  ret fm

# toClip is a built-in function that turn frames into a clip 
# apply "changeColor" to part of clip
newClip = clip(:5.6) + toClip(map(lambda x->changeColor(x),
				clip(5.6:15.6).frames())) + 
            clip(15.6:)
\end{lstlisting}

\subsection{Analysis}
Here we show how \textit{Yo} performs analytical editing.
The below expression performs the following:
\begin{enumerate}[label=(\alph*)]
\item Filter the clips and only keep those shorter than 10 seconds;
\item Scale the clips to 720x360
\item Trim off the first and the last second in each clip;
\item Concatenate these clips into a video.
\end{enumerate}

\begin{lstlisting}[caption=Calculate based on videos]
lapply(+, Clip(), 
  [c(1.0:-1.0).scale(720) for c in clips if c.duration < 10.0]
)
\end{lstlisting}

The analytical editing becomes quite handy when we want to extract clip pieces of some interesting features from a long, everlasting (think of a security camera) video. For example, the following statement extract all clips with Yo's appearance:
\begin{lstlisting}[caption=Identify Yo's face :)]
face = readFrame("yo_face.png")
yo_appearance = lapply(+, Clip(), 
   [fm for fm in clips.frames() if imageMatch(fm, face) > 0.95])
)
\end{lstlisting}


\section{Syntax}
\subsection{Types}
Like a scripting language, users do not need explicit type declaration for variables. The equal sign $=$ is used to assign values to variables. Types of variables are inferred and could be overwritten.\\ 
Basic data types: 
\begin{enumerate}
\item \textit{int}: signed integers
\item \textit{double}: floating point real values 
\item \textit{bool}: boolean 
\item \textit{string}: a contiguous set of characters
\end{enumerate}
Composite data types:
\begin{enumerate}
\item \textit{array}: holds a sequence of elements of the same type
\item \textit{tuple}: holds a sequence of elements, immutable
\item \textit{struct}: a user-defined prototype for an object that defines a set of attributes including variables and methods
\end{enumerate}

\begin{lstlisting}[caption=TypeExample]
answer  = 42           # int
endtime = 7.5          # double
subtitle = "Yo,world"  # string
criteria = (a > 2)     # boolean
clips = []             # array
color = (255,136,23)   # tuple
clip = Clip()          # struct
\end{lstlisting}

\subsection{Operators}
\textit{Yo} provides with operators as shown in Table\ref{tab:operator}.

\begin{table}[h!]
  \begin{center}
    \caption{Operators and Notations in Yo.}
    \label{tab:operator}
    \begin{tabular}{c|l}
      \# & start of comment line\\
      \hline
      \text{\#\{} & start of multi-line comment\\
      \hline
      \text{\#\}} & end of multi-line comment\\
      \hline
      \text{+} & \begin{tabular}[x]{@{}l@{}} add operator\\ concatenate clips\end{tabular}\\
      \hline
      \text{-} & subtract \& negate operator\\
      \hline
      \text{*} & multiply operator\\
      \hline
      \text{/} & divide operator\\
      \hline
      \text{\%} &\begin{tabular}[x]{@{}l@{}} mod operator \\ format output specifier\end{tabular}\\
      \hline
      \text{\&}  &\begin{tabular}[x]{@{}l@{}}same layer operator, a \& b set the z-index of layer in clip b\\ equals to the one of clip a \\\end{tabular}\\
      \hline
      \text{\^{}} \text{@} & \begin{tabular}[x]{@{}l@{}} above layer operator, a \^{} b \text{@} c set the z-index of layer in clip b\\ larger than the one of clip a, with a offset of c second \\ \end{tabular}\\
      \hline
      \text{\&\&} & and operator \\
      \hline
      \text{||} & or operator \\
      \hline
      \text{!} & not operator \\
      \hline
      \text{<} & less-than operator \\
      \hline
      \text{<=} & less-than-or-equal operator \\
      \hline
      \text{==} & equal operator \\
      \hline
      \text{>=} & greater-than-or-equal operator \\
      \hline
      \text{>} & greater-than operator \\
      \hline
      \text{=} & assign operator \\
      \hline
      \text{<-} & \begin{tabular}[x]{@{}l@{}} list comprehension generator \\ single assignment operator \end{tabular} \\
      \hline
      \text{->} & lambda function definition operator\\
      \hline
      \text{.} & call member or function in struct operator\\
      \hline
      \text{:} & list slice operator \\ 
      \hline
      $\sim$ & inference operator \\
      \hline
      \text{"}, \text{'} & string construction operator\\
      \hline
      \text{[ ]} & array construction operator\\ 
      \hline
      \text{( )} & \begin{tabular}[x]{@{}l@{}} tuple construction operator \\ clip time access operator \end{tabular}\\
      \hline
      \text{,} & separator
    \end{tabular}
  \end{center}
\end{table}


\subsection{Control Flow}
\textit{Yo} supports basic control flow. To help understand the code, the \textit{Yo} code snippet is followed by C++ code that achieves the same effect.

\begin{lstlisting}[caption=\textit{Yo} Control Flow Example]
# conditional statement
if clip.time > 10
  log(clip.time)
   
# cascading for-loop
for i <- 1 to 10, j <- 1 to 10, i + j == 10
  log("%d+%d=%d\n", i, j, i + j)
    
s = 0
for i <- 10 downto 1 by -1, i != 2, x <- a[i]
  s += x
    
# suffix if/while/for
log("Yo world") if length > 100
fun1() while a > b
a[i] = 0 for i <- 1 to 10, i % 2 == 0
\end{lstlisting}

\begin{lstlisting}[caption=C++ Control Flow Example]
int main() {
  if (clip.time > 10){
    printf("%d",clip.time);
  }
  for (auto i = 1; i <= 10; ++i) {
    for (auto j = 1; j <= 10; ++j) {
      if (i + j == 10) {
        printf("%d+%d=%d\n", i, j, i + j);
      }
    }
  }
    auto sum = 0;
    for (auto i = 10; i >= 1; i = i + -1) {
      if (i != 2) {
        for (auto x : a[i]) {
          sum += x;
        }
      }
    }
    if (length > 100) {
      printf("Yo world");
    }
    while (a > b) {
      fun1();
    }
    for (auto i = 1; i <= 10; ++i) {
      if (i % 2 == 0) {
        a[i] = 0;
      }
    }
}
\end{lstlisting}

\subsection{Functions}
\textit{Yo} functions are defined starting with keyword \texttt{func}. Lambda functions are also supported.
\begin{lstlisting}[caption=Function Example]
# function declaration
func longerTime (a, b)
  if a.duration > b.duration
    return a.duration
  else
    return b.duration

# function calls
longerTime(clip1,clip2)

# lambda functions
pixels = map(lambda x -> x.r=200 && x.b=20, selected_pixels)
\end{lstlisting}

\subsection{IO}
\textit{Yo} load video, audio and image files on the drive, render the  timeline and output edited video. To debug and log, a log function that dumps standard output stream is provided.
\begin{lstlisting}[caption=IO Example]
# read all images in a directory "wd"
frames = [readFrame(f) for f in wd if f.endwith(".png")]

# read all videos in directory "wd"
clips = [readVideo(f) for f in wd if f.endwith(".avi")]

# log file name to stdout
log(clips[0].filename)

# output final movie
saveClip("myYo.webm",final_cut)
\end{lstlisting}
\end{document}